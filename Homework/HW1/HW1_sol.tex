%!TEX program = xelatex
\documentclass[10pt]{article}
\usepackage{xeCJK}
\usepackage{amsmath,amsfonts,amssymb}
\usepackage{color}
\usepackage{tcolorbox}
\usepackage[hidelinks,unicode,psdextra]{hyperref}
\usepackage{tikz-cd}
\usepackage{tikz}
\usetikzlibrary{graphs}
\usetikzlibrary{cd}
\usepackage{graphicx}
\usepackage{amsthm}
\usepackage{geometry}
\usepackage{enumerate,mathdots}
\usepackage[ruled,vlined]{algorithm2e}
\tcbuselibrary{breakable}
\geometry{a4paper,scale=0.8}
\newcommand{\mr}[1]{\mathrm{#1}}
\newcommand{\mb}[1]{\mathbb{#1}}
\newcommand{\ds}{\displaystyle}
\newcommand{\me}{\mathrm{e}}
\newcommand{\mi}{\mathrm{i}}
\newcommand{\md}{\mathrm{d}}
\newcommand{\mc}{\mathcal}
\newcommand{\ms}{\mathscr}
\newcommand{\jian}{\overrightarrow}
\addtocounter{MaxMatrixCols}{10}
\DeclareSymbolFont{lettersA}{U}{pxmia}{m}{it}
\DeclareMathSymbol{\piup}{\mathord}{lettersA}{"19}
\allowdisplaybreaks[4]
\title{Homework 1}
\author{PB20000112\quad 黄天一}
\date{}
\begin{document}
	\maketitle
	\begin{enumerate}
		\begin{tcolorbox}
			\item 假定$f(n)$和$g(n)$都是渐进非负函数, 判断下列等式或陈述是否一定是正确的, 并简要解释你的答案.
			\begin{enumerate}
				\item $f(n)=O(f(n)^2).$
				\item $f(n)+g(n)=\Theta(\max(f(n),g(n))).$
				\item $f(n)+O(f(n))=\Theta(f(n)).$
				\item if $f(n)=\Omega(g(n))$, then $g(n)=o(f(n))$.
			\end{enumerate}
		\end{tcolorbox}
		\begin{tcolorbox}[colback = red!20!white]
			\textit{Solution.}\ 
			\begin{enumerate}
				\item 错误. 令$f(n)=\dfrac{1}{n}$, 则$\lim\limits_{n\to+\infty}\dfrac{f(n)}{f(n)^2}=\lim\limits_{n\to+\infty}n=+\infty$, 即$f(n)=\omega(f(n)^2)$, 与$f(n)=O(f(n)^2)$相悖.
				\item 正确. 由于
				$$(f(n)+g(n))-\max(f(n),g(n))=\min(f(n),g(n))\geqslant 0;$$
				$$\max(f(n),g(n))-\dfrac{f(n)+g(n)}{2}=\dfrac{\max(f(n),g(n))-\min(f(n),g(n))}{2}\geqslant 0,$$
				因此$\forall n\in\mb{N}$, $\dfrac{f(n)+g(n)}{2}\leqslant\max(f(n),g(n))\leqslant f(n)+g(n)$, 即$f(n)+g(n)=\Theta(\max$ $(f(n),g(n))$.
				\item 正确. 由定义得$\exists n_0\in\mb{N}$, $\forall n\geqslant n_0$, $\exists c\in\mb{R}^+$, s.t. $0\leqslant O(f(n))\leqslant cf(n)$. 因此$f(n)\leqslant f(n)+O(f(n))\leqslant (c+1)f(n)$, 即$f(n)+O(f(n))=\Theta(f(n))$.
				\item 错误. 设$f(n)=2n$, $g(n)=n$, 则$\forall n\in\mb{N},0<g(n)<f(n)\Rightarrow f(n)=\Omega(g(n))$, 但是$\lim\limits_{n\to\infty}\dfrac{g(n)}{f(n)}=\dfrac{1}{2}\neq 0\Rightarrow g(n)\neq o(f(n))$.
			\end{enumerate}
		\end{tcolorbox}
		\begin{tcolorbox}
			\item 时间复杂度.
			\begin{enumerate}
				\item 证明$\lg(n!)=\Theta(n\lg n)$, 并证明$n!=\omega(2^n)$且$n!=o(n^n)$.
				\item 使用代入法证明$T(n)=T(\lceil n/2\rceil)+1$的解为$O(\lg n)$.
				\item 对递归式$T(n)=T(n-a)+T(a)+cn$, 利用递归树给出一个渐进紧确解, 其中$a\geqslant 1$和$c>0$为常数.
				\item 主方法能应用于递归式$T(n)=4T(n/2)+n^2\lg n$吗? 请说明为什么可以或者为什么不可以. 给出这个递归式的一个渐进上界.
			\end{enumerate}
		\end{tcolorbox}
		\begin{tcolorbox}[colback=red!20!white]
			\textit{Solution.}\ 
			\begin{enumerate}
				\item 首先有$\lg(n!)=\sum\limits_{k=1}^n\lg k\leqslant\sum\limits_{k=1}^n\lg n=n\lg n$, 故$\lg(n!)=O(n\lg n)$.  注意到$\lg(n!)$的部分和
				$$\sum\limits_{k=\lfloor n/2\rfloor}^n\lg k\geqslant\sum\limits_{k=\lfloor n/2\rfloor}^n\lg\left\lfloor\dfrac{n}{2}\right\rfloor\geqslant\dfrac{n}{2}\lg\left\lfloor\dfrac{n}{2}\right\rfloor=\Theta(n\lg n),$$
				因此$\lg(n!)=\Omega(n\lg n)$. 综上所述, $\lg(n!)=\Theta(n\lg n).$
				
				应用Stirling公式: $n!=\sqrt{2\pi n}\left(\dfrac{n}{\me}\right)^n\left(1+\Theta\left(\dfrac{1}{n}\right)\right),$ 可得
				$$\lim_{n\to+\infty}\dfrac{n!}{2^n}=\lim_{n\to+\infty}\sqrt{2\pi n}\left(\dfrac{n}{2\me}\right)^n\left(1+\Theta\left(\dfrac{1}{n}\right)\right)=+\infty,$$
				$$\lim_{n\to+\infty}\dfrac{n!}{n^n}=\lim_{n\to+\infty}\dfrac{\sqrt{2\pi n}}{\me^n}\left(1+\Theta\left(\dfrac{1}{n}\right)\right)=0,$$
				因此$n!=\omega(2^n)$, $n!=o(n^n)$.
				\item 设$T(n)\leqslant a\lg n+b$, $a>0,b\in\mb{R}$为待定参数. 应用归纳法, 可得
				$$T(n)=T(\lceil n/2\rceil)+1\leqslant a\lg(\lceil n/2\rceil)+b+1\leqslant a\lg n+b-(a\lg\dfrac{3}{2}-1),$$
				取$a=\dfrac{1}{2}$, 则$a\lg\dfrac{3}{2}-1=\dfrac{1}{2}\lg\dfrac{3}{8}<0$, 此时$T(n)<a\lg n+b$, 归纳成立. 因此$T(n)\leqslant\dfrac{1}{2}\lg n+O(1)$, 即$T(n)=O(\lg n)$.
				\item 题设递归式的决策树如Figure \ref{fig:Recursion_Tree}所示, 其中$r=n(\mr{mod}\;a)$. 决策树的高度为$h=\lceil n/a\rceil$, 叶子结点为一个$T(r)$和$h$个$T(a)$. 因此
				$$T(n)=hT(a)+T(r)+\sum\limits_{k=0}^{h-2}c(n-ka)=cn\lceil n/a\rceil+\Theta(n)=\Theta(n^2).$$
				\item 不能应用Master Theorem. 注意到$n^2\lg n=\Omega(n^2)$, 但是 对于$\forall\varepsilon>0$, $\lg n=o(n^{\varepsilon})$, 因此不存在$\epsilon>0$, s.t. $n^2\lg n=O(n^{2-\epsilon})$或$n^2\lg n=\Omega(n^{2+\epsilon})$, 不满足Master Theorem的条件.
				
				猜测: 题设递推式的一个渐进上界为$n^2\lg^2n$, 设$T(n)\leqslant an^2\lg^2n$, $a>0$. 应用归纳法, 可得当$n\geqslant 2$时, 
				\begin{align*}
					T(n)&=4T(n/2)+n^2\lg n\\
					&\leqslant a^2n^2\lg^2(n/2)+n^2\lg n\\
					&=a^2n^2\lg^2n+a^2n^2+(1-2a^2)n^2\lg n\\
					&\leqslant a^2n^2\lg^2n+(1-a^2)n^2\lg n.
				\end{align*}
				令$a=1$, 则$T(n)\leqslant n^2\lg^2n$, 归纳成立. 因此$T(n)=O(n^2\lg^2n)$.
			\end{enumerate}
		\end{tcolorbox}
		\begin{figure}[h]
			\centering
			\begin{tikzpicture}
				\node[circle, fill=black, scale=0.2] (R) at (0,0){};
				\coordinate[label=above:$T(n)$] (R1) at (0,0);
				\node[circle, fill=black, scale=0.2] (T1) at (-3,-2){};
				\coordinate[label=left:$T(n-a)$] (T11) at (T1);
				\node[circle, fill=black, scale=0.2] (T2) at (3,-2){};
				\coordinate[label=right:$T(a)$] (T21) at (T2);
				\node (Z1) at (5,-2){$cn$};
				\node[circle, fill=black, scale=0.2] (W1) at (-4,-4){};
				\coordinate[label=left:$T(n-2a)$] (W11) at (W1);
				\node[circle, fill=black, scale=0.2] (W2) at (-2,-4){};
				\coordinate[label=right:$T(a)$] (W21) at (W2);
				\node (Z2) at (5,-4){$c(n-a)$};
				\node (Z3) at (5,-5){$\vdots$};
				\node (X) at (-4,-5){$\vdots$};
				\node[circle, fill=black, scale=0.2] (X2) at (-4,-6){};
				\node[circle, fill=black, scale=0.2] (L1) at (-4.5,-7){};
				\coordinate[label=left:$T(r)$] (L11) at (L1);
				\node[circle, fill=black, scale=0.2] (L2) at (-3.5,-7){};
				\coordinate[label=right:$T(a)$] (L21) at (L2);
				\node (Z4) at (5,-7){$c(r+a)$};
				\draw (R1) to (T1);
				\draw (R1) to (T2);
				\draw (T1) to (W1);
				\draw (T1) to (W2);
				\draw (X2) to (L1);
				\draw (X2) to (L2);
			\end{tikzpicture}
			\caption{$T(n)=T(n-a)+T(a)+cn$的决策树.}
			\label{fig:Recursion_Tree}
		\end{figure}
		\begin{tcolorbox}
			\item 对下列递归式, 使用主方法求出渐进紧确解.
			\begin{enumerate}
				\item $T(n)=2T(n/4)+\sqrt{n}.$
				\item $T(n)=2T(n/4)+n^2.$
			\end{enumerate}
		\end{tcolorbox}
		\begin{tcolorbox}[colback=red!20!white]
			\textit{Solution.} 将题设递归式均视为$T(n)=aT(n/b)+f(n)$的形式. 
			\begin{enumerate}
				\item $f(n)=\sqrt{n}$, $a=2$, $b=4$. 因此$f(n)=\Theta(n^{\log_b a})$. 应用Master Theorem可得$T(n)=\Theta(n^{\log_b a}\lg n)=\Theta(\sqrt{n}\lg n)$.
				\item $f(n)=n^2$, $a=2$, $b=4$. 因此对于$\varepsilon=1$有$f(n)=\Omega(n^{\log_b a+1})$. 又因为$af(n/b)=n^2/8<f(n)$, 满足条件. 应用Master Theorem可得$T(n)=\Theta(n^2)$.
			\end{enumerate}
		\end{tcolorbox}
		\begin{tcolorbox}
			\item 考虑以下查找问题:\\
			\textbf{输入:} $n$个数的一个序列$A=a_1,a_2,\cdots,a_n$和一个值$v$.\\
			\textbf{输出:} 下标$i$使得$v=A[i]$或者当$v$不在$A$中出现时, $v$为特殊值$NIL$.
			\begin{enumerate}
				\item 写出\textbf{线性查找}的伪代码, 它扫描整个序列来查找$v$. 使用一个Loop Invariant(循环不变式)来证明你的算法是正确的.
				\item 假定$v$等可能地为数组的任意元素, 平均需要检查序列的多少元素? 最坏情况又如何呢? 用$\Theta$记号给出线性查找的平均情况和最坏运行时间.
			\end{enumerate}
		\end{tcolorbox}
		\begin{algorithm}[h]
			\caption{Linear Search.}
			\label{alg:algorithm1}
			\KwIn{a sequence of $n$ numbers $A=a_1,a_2,\cdots,a_n$ and a value $v$.}
			\KwOut{subscript $i$ such that $v=A[i]$ or when $v$ does not occur in $A$, $v$ is the special value $NIL$.}  
			\BlankLine
			\For{$i=1$ \textnormal{\textbf{to}} $A.length$}{
				\If{$A[i]==v$}{
					\textbf{return} $i$;
				}
			}
			\textbf{return} $NIL$;
		\end{algorithm}
		\begin{tcolorbox}[colback=red!20!white, breakable]
			\textit{Solution.}\ 
			\begin{enumerate}
				\item 线性查找算法如Algorithm \ref{alg:algorithm1}所示. 给定一个Loop Invariant: 在Algorithm \ref{alg:algorithm1}的第$i$次迭代开始前, 算法返回$i-1$并终止或子列$A[1..i-1]$不包含元素$v$. 证明如下:\\
				\textbf{初始化:} 在迭代开始之前, 子列$A[1..0]$为空, 显然不包含元素$v$;\\
				\textbf{保持:} 在第$i$此迭代开始前,  设循环不变式为真. 若算法终止并返回$i-1$, 则迭代结束;  若$A[1..i-1]$不包含元素$v$, 则进行第$i$次迭代. 若$A[i]=v$, 则算法返回下标$i$并终止代; 若$A[i]\neq v$, 则此时$A[1..i]$不包含元素$v$, 循环不变式成立, 迭代继续.\\
				\textbf{终止:} 算法终止时, 若$i\leqslant n$, 则算法返回下标$i$, 此时$v$在序列$A$中; 若$i=n+1$, 则根据循环不变式, $A[1..i-1]=A[1..n]$不包含元素$v$, 即序列$A$不包含$v$, 查找失败, 返回特殊值$NIL$.\\
				综上所述, 线性查找算法Algorithm \ref{alg:algorithm1}是正确的.
				\item 设$1\leqslant i\leqslant n$, 若$v$在序列$A$中,  则$A[i]=v$的概率为$p_i=\dfrac{1}{n}$, 此时需检查元素$A[1..i]$, 共$i$个元素.  设检查序列元素的个数为$X$, 则期望为
				$$\mb{E}[X]=\sum\limits_{i=1}^nip_i=\frac{1}{n}\sum\limits_{i=1}^ni=\frac{n+1}{2}=\Theta(n/2).$$
				最坏情况下, $A[n]=v$或$v$不在序列$A$中, 此时需要检查序列中的全部元素, 即需要检查$n=\Theta(n)$次.
			\end{enumerate}
		\end{tcolorbox}
		\begin{tcolorbox}
			\item 堆排序:\\
			对于一个按升序排列的包含$n$个元素的有序数组$A$来说, HEAPSORT的时间复杂度是多少? 如果$A$是降序的呢? 请简要分析并给出结果.
		\end{tcolorbox}
		\begin{tcolorbox}[colback=red!20!white,breakable]
			\textit{Solution.}\ 
			\begin{enumerate}
				\item $A$按升序排列. 则对于$1\leqslant i\leqslant\lfloor n/2\rfloor$, 均有$A[i]\leqslant A[j]$, $i<j$. 因此在建堆过程中的每次移动中, 元素$A[i]$都需要从当前位置移动到叶子处(因为此前只移动了下标大于$i$的元素), 移动次数为$n_i=\lfloor\lg n\rfloor-\lfloor\lg i\rfloor$. 元素比较时间为$\lfloor n/2\rfloor=\Theta(n)$, 因此
				$$T(n)=\Theta(n)+\sum\limits_{i=1}^{\lfloor n/2\rfloor}n_i=\sum\limits_{i=1}^{\lfloor n/2\rfloor}(\lfloor\lg n\rfloor-\lfloor\lg i\rfloor)=\Theta\left(\sum\limits_{i=1}^{\lfloor n/2\rfloor}\lg(n/i)\right)=\Theta(\lg(n!))=\Theta(n\lg n).$$
				其中最后一个等式运用了2(a)的结论.
				\item $A$按降序排列. 则对于$1\leqslant i\leqslant\lfloor n/2\rfloor$, 均有$A[i]\geqslant A[2i],A[2i+1]$, 此时$A$即为MAXHEAP. 因此建堆过程中只需要比较元素, 不需要移动元素. 因此$T(n)=\lfloor n/2\rfloor=\Theta(n)$.
			\end{enumerate}
		\end{tcolorbox}
		\begin{tcolorbox}
			\item 快速排序:
			\begin{enumerate}
				\item 假设快速排序的每一层所做的划分比例都是$1-\alpha:\alpha$, 其中$0<\alpha\leqslant 1/2$且是一个常数. 试证明: 在相应的递归树中, 叶结点的最小深度大约是$-\lg n/\lg\alpha$, 最大深度大约是$-\lg n/\lg(1-\alpha)$(无需考虑舍入问题).
				\item 试证明: 在一个随机输入数组上, 对于任何常数$0<\alpha\leqslant 1/2$, PARTITION产生比$1-\alpha:\alpha$更平衡的划分的概率约为$1-2\alpha$.
			\end{enumerate}
		\end{tcolorbox}
		\begin{tcolorbox}[colback=red!20!white,breakable]
			\begin{proof}
				\ 
				\begin{enumerate}
					\item PARTITION算法的时间复杂度为$\Theta(n)$. 设QUICKSORT的时间复杂度为$T(n)$, 则有$T(n)=T(\alpha n)+T((1-\alpha)n)+\Theta(n)$. 设递归树中$T(n)$的LEFTCHILD结点为$T(\alpha n)$, RIGHTCHILD结点为$T((1-\alpha)n)$. 由$0<\alpha\leqslant 1/2$可得, 从ROOT开始始终沿LEFTCHILD方向的路径上, CHILD结点对应的时间复杂度的下降速率最快$(1/\alpha)$, 因此该路径上的叶结点具有最小深度$h_{\min}=\log_{1/\alpha}n=-\lg n/\lg\alpha$; 从ROOT开始始终沿RIGHTCHILD方向的路径上, CHILD结点对应的时间复杂度的下降速率最慢$(1/(1-\alpha))$, 因此该路径上的叶结点具有最大深度$h_{\max}=\log_{1/(1-\alpha)}n=-\lg n/\lg(1-\alpha)$.
					\item 递归树叶结点的深度的极差$\Delta h=h_{\max}-h_{\min}$越小, 递归树对应的PARTITION的划分越平衡. 由于
					$$\Delta h=\frac{\lg n}{\lg\alpha}-\frac{\lg n}{\lg(1-\alpha)}=\lg n\left(\frac{1}{\lg\alpha}-\frac{1}{\lg(1-\alpha)}\right),$$
					因为$\lg\alpha$, $-\dfrac{1}{\lg(1-\alpha)}$在区间$(0,1/2]$上均为关于$\alpha$的递减函数, 因此$\alpha$越大, PARTITION越平衡. 设对于$\alpha\in(0,1/2]$, PARTITION产生比$1-\alpha:\alpha$更平衡的划分的概率为$p$, 则$p=\dfrac{1/2-\alpha}{1/2}=1-2\alpha$.
				\end{enumerate}
			\end{proof}
		\end{tcolorbox}
	\end{enumerate}
\end{document}